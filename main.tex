\documentclass[%
reprint,
superscriptaddress,
%linenumbers,
amsmath,amssymb,
aps,
prb,
showkeys,
]{revtex4-2}
\usepackage[T1]{fontenc}
%\usepackage{bera}
\usepackage{color}
\usepackage{soul}
%\usepackage{draftwatermark}
\usepackage{graphicx}% Include figure files
\usepackage{dcolumn}% Align table columns on decimal point
\usepackage{bm}% bold math
\usepackage{physics}
\usepackage{amssymb}
\usepackage[sort&compress]{natbib}
\usepackage{hyperref}
\usepackage[capitalize]{cleveref}
\usepackage{color}
\newcommand{\blue}[1]{\textcolor{blue}{#1}}
\newcommand{\red}[1]{\textcolor{red}{#1}}
\usepackage{soul}
\usepackage{todonotes}
\usepackage[normalem]{ulem}

\begin{document}

\preprint{APS/123-QED}

\title{Advancements in Quantum Chimera States: A Comprehensive Review}% Force line breaks with \\

\author{Mahbub Rahaman}
\email{mahbub.phys@gmail.com}
\thanks{Primary \& Corresponding author}
\affiliation{Department of Physics, The University of Burdwan, Golapbag, Bardhaman - 713104, India}
\author{Analabha Roy}
\email{daneel@utexas.edu}
\thanks{Co-corresponding author}
\affiliation{Department of Physics, The University of Burdwan, Golapbag, Bardhaman - 713104, India}


\begin{abstract}
	Quantum many-body systems have a tendency to thermalize, leading to the loss of memory of their initial states. However, in clean periodically driven quantum integrable systems, it is possible to preserve the initial state of the system for an extended duration under specific freezing conditions. This phenomenon, known as dynamical many-body freezing. A similar freezing has also been observed in a non-integrable spin-1/2 system called the Lipkin-Meshkov-Glick (LMG) model. At the freezing point, the heating of the LMG system is effectively suppressed, violating the eigenstate thermalization hypothesis and manifesting localization in the many-body system. This phenomenon is referred to as dynamical many-body localization (DMBL). DMBL is validated through the inverse participation ratio of the quasi-stationary Floquet states [1]. Leveraging these findings, we have discovered a chimeralike state in which a spin-1/2 system exhibits coexisting a discrete time crystal phase and DMBL phase. A time crystal is a phase of matter that breaks time translation symmetry. This chimeralike state persists for a significant duration and remains stable even at the thermodynamic limit [2]. Our results provide a novel perspective on the intricate interplay between time crystals and quantum many-body systems, with potential implications for the advancement of cutting-edge quantum technologies.
\end{abstract}

%\keywords{Dynamical localization, Thermalization, Phase Crossover}
\maketitle

\section{Introduction}
A flat band protocol manifests all the energy levels of a system to be degenerate, mostly at zero energy \red{(cite refs)}. This is interesting because it can lead to a variety of exotic phenomena, such as topological phases, fractionalization, and many-body localization  \red{(cite refs)}.\\ 
\vskip 1cm
	keep writing more on introduction here \dots\dots
\section{The model and system dynamics}
 The generic flat band protocol involves the incorporation of two time dependent drives that are engineered in such a way that the dynamics of the system gets nullified, resulting in a flat band and consequently establishes the localization in the system. Now we propose a one dimensional spin-1/2 system, consisting of $N$ sites, driven by two time dependent drives protocol, which is given by,
\begin{align}
    \hat{H}(t) &= \hat{H}_1(t) + \hat{H}_2(t),\\
    \hat{H}_1 &= \lambda_s (1-\epsilon) \hbar \sum_i \hat{\sigma}^x_i, &\quad T_1 = 0<t \le T/2, \\
            &= 0, &\quad T_2 = T/2<t \le T, \\
    \hat{H}_2 &= 0, &\quad T_1 = 0<t \le T/2, \\
     &= \hat{H}_{fb}(t), &\quad T_2  = T/2<t \le T.
\end{align}
Here, $\hbar$ is the Planck's constant, $T_1 = T_2 =T/2$, while $T = 2\pi/\omega$, being the time period with frequency $\omega$ of the drive protocol. Moreover, $\hat{\sigma}^{\mu=x,y,z}_i$ are the Pauli matrices at site $i$. Additionally, we consider $\lambda_s T_1 = \pi/2$ (therefore, $\lambda_s=\omega/2$), such that the Hamiltonian $\hat{H}_1$, consisting of $\displaystyle \sum_i \hat{\sigma}^x_i$ component, flips spins at site `$i$' during the $T_1$ time cycle. The spin rotational error ($\epsilon$) is initially considered to be zero, however we have introduced it latter section (\red{cite section}) in order to investigate the stability of the proposed model. The Hamiltonian $\hat{H}_{fb}(t)$ manifests flat band in the system during $T_2$ cycle of each of the periods, given as,
\begin{align}
    \hat{H}_{fb}(t) &= \lambda_1 (t) J \hbar\sum_{i=1}^{N} \hat{\sigma}_i^x \hat{\sigma}_{i+1}^x + \lambda_2 (t) h_z \hbar \sum_i \hat{\sigma}_i^z\\
     &= \mathrm{sgn}[\cos(n\omega t)] J \hbar\sum_{i=1}^{N} \hat{\sigma}_i^x \hat{\sigma}_{i+1}^x\nonumber\\
	 & \hspace{1cm} + \mathrm{sgn}[\cos((2n+1)\omega t)] h_z \hbar\sum_i \hat{\sigma}_i^z ,
\end{align}
where $n$ is an integer ($n = 1, 2, 3, \dots$), $J$ is the nearest neighbor spin coupling strength and $h_z$ is the amplitude of the transverse field. The time dependent functions $\lambda_1(t)$ and $\lambda_2(t)$ are defined as,
\begin{align}
    \lambda_1(t) &= \lambda_0 \;\mathrm{sgn}[\cos(n\omega t)],\\
    \lambda_2(t) &= \omega_0 - \omega_1 \;\mathrm{sgn}[\cos((2n+1)\omega t)].
\end{align}
 In numerical investigations we have considered $n=1$, $\omega_0 = 0$, $\hbar =1$, $\lambda_1 = 1$, $J=1$, $h_z=1$ and $\omega_0 = 0.5$ as shown in Fig.~\ref{fig:flatband_drive_protocol}.
\begin{figure}[t]
    \centering
    \includegraphics[width=7.5cm]{figs/drive_protocol.pdf}
    \caption{Three time dependent drives $\lambda_{1,2,s}(t)$ protocol for the proposed model. The first time interval, $T_1$, consists of a spin-flip drive, while the second time interval, $T_2$, consists of a flat band protocol. We have considered the primary drive frequency $\omega = 2$, consequent amplitude $\lambda_s = 1$ and flat band drive amplitudes $\lambda_1 = 1$, $\lambda_2 = 0.5$.}
    \label{fig:flatband_drive_protocol}
\end{figure}
These two time dependent functions with different amplitudes and frequencies are engineered in such a way that the system dynamics is nullified, resulting in a flat band. Consequently, during the second time interval, the system becomes localized to its initial state of second time interval. Furthermore, there is no spin-spin interaction term present during the first time interval,  If the system is evolved for another time period, the flipped spins are reversed, resulting in breaking the time-reversal symmetry and giving rise to the phenomenon known as a discrete time crystal (DTC) phase.

\textit{Floquet engineering:} Due to the time-dependency of the Hamiltonian, the conservation of energy is violated. Therefore, it is convenient to introduce the Floquet theory, which simplifies the problem by introducing concept of the  quasi-stationary states and energy eigenvalues. The system propagator is derived from the Schr\"odinger equation, $\displaystyle i \hbar \partial_t \hat{U}(t) = \hat{H}(t) \hat{U}(t)$. The time evolution of the system is governed by the Floquet Hamiltonian, which is a time-independent operator constructed from the time-dependent Hamiltonian of the system.
The floquet operator is defined as,
\begin{equation}
    \hat{\mathcal{F}} = \exp(-\frac{i}{\hbar}\hat{H}_F T),
\end{equation}
here, $\hat{H}_F$ is the Floquet Hamiltonian, and $T$ is the time period of the drive. The Floquet Hamiltonian is derived from the time-dependent Hamiltonian of the system, $\hat{H}(t)$, as, $\displaystyle  \hat{H}_F = \left(\hat{H}(t) -i\hbar \partialderivative{t}\right)$, evaluated at strobed time interval, $t=nT\; ;n=1,2,3,\dots$.

Considering the Floquet operator as $\hat{\mathcal{F}}$, the effective Floquet Hamiltonian ($\hat{H}^{\mathrm{eff}}$) for the proposed DTC at $2T$ time is given by,
\begin{align}
    \hat{\mathcal{F}} &= \exp(- \frac{i}{\hbar} \hat{H}^{\mathrm{eff}} 2T)\nonumber\\
    & = \exp(-\frac{i}{\hbar}\hat{H}_2 T_2) \exp(-\frac{i}{\hbar}\hat{H}_1 T_1)\nonumber\\ 
	&\hspace{2cm} \exp(-\frac{i}{\hbar}\hat{H}_2 T_2) \exp(-\frac{i}{\hbar}\hat{H}_1 T_1)
\end{align}

\section{\label{sec:level7}Conclusion and Outlook}
Write Conclusion and Outlook here\dots\dots

\begin{acknowledgments}
MR acknowledges The University of Burdwan for support via the state-funded fellowship. AR acknowledges support from the University Grants Commission (UGC) of India, Grant No. F.30-425/2018(BSR), as well as from the Science and Engineering Research Board (SERB), Grant No. CRG/20l8/004002.
\end{acknowledgments}

%\nocite{*}
\bibliographystyle{apsrev4-2}
\bibliography{main}% Produces the bibliography via BibTeX.

\end{document}
%
% ****** End of file apssamp.tex ******
